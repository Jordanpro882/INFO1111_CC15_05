\documentclass[a4paper, 11pt]{report}
\usepackage{blindtext}
\usepackage[T1]{fontenc}
\usepackage[utf8]{inputenc}
\usepackage{titlesec}
\usepackage{fancyhdr}
\usepackage{geometry}
\usepackage{fix-cm}
\usepackage[hidelinks]{hyperref}
\usepackage{graphicx}

\usepackage[english]{babel}

\geometry{ margin=30mm }
\counterwithin{subsection}{section}
\renewcommand\thesection{\arabic{section}.}
\renewcommand\thesubsection{\thesection\arabic{subsection}.}
\usepackage{tocloft}
\renewcommand{\cftchapleader}{\cftdotfill{\cftdotsep}}
\renewcommand{\cftsecleader}{\cftdotfill{\cftdotsep}}
\setlength{\cftsecindent}{2.2em}
\setlength{\cftsubsecindent}{4.2em}
\setlength{\cftsecnumwidth}{2em}
\setlength{\cftsubsecnumwidth}{2.5em}


\begin{document}
\titleformat{\section}
{\normalfont\fontsize{15}{0}\bfseries}{\thesection}{1em}{}
\titlespacing{\section}{0cm}{0.5cm}{0.15cm}
\titleformat{\subsection}
{\normalfont\fontsize{13}{0}\bfseries}{\thesubsection}{0.5em}{}
\titlespacing{\section}{0cm}{0.5cm}{0.15cm}

%=======================================================================================

% #########################
% IMPORTANT - Add student names here!
% e.g. \newcommand{\stud1}{LOWE, David}
\newcommand{\studA}{{BLIGHT, Matthew}}
\newcommand{\studB}{{FAMNAME2, givenName2}}
\newcommand{\studC}{{TRAN, Jordan}}
\newcommand{\studD}{{FAMNAME4, givenName4}}
%
% IMPORTANT - Then give your SIDs
\newcommand{\sidA}{{490423105}}
\newcommand{\sidB}{{01234567}}
\newcommand{\sidC}{{530503851}}
\newcommand{\sidD}{{01234567}}
%
% IMPORTANT - And then update which major each student will focus on
\newcommand{\majA}{{Computer Science}}
\newcommand{\majB}{{Data Science}}
\newcommand{\majC}{{SW Development}}
\newcommand{\majD}{{Cyber Security}}
% #########################


\pagenumbering{Alph}
\begin{titlepage}
\begin{flushright}
\includegraphics[width=4cm]{USyd}\\[2cm]
\end{flushright}
\center 
\textbf{\huge INFO1111: Computing 1A Professionalism}\\[0.75cm]
\textbf{\huge 2023 Semester 1}\\[2cm]
\textbf{\huge Skills: Team Project Report}\\[3cm]

\textbf{\huge Submission number: 1}\\[0.75cm]
\textbf{Github link: https://github.sydney.edu.au/mbli9416/INFO1111\_CC15\_05}\\[0.75cm]
\textbf{\huge Team Members:}\\[0.75cm]

\begin{tabular}{|p{0.25\textwidth}|p{0.13\textwidth}|p{0.12\textwidth}|p{0.12\textwidth}|p{0.22\textwidth}|}
	\hline
	Name & Student ID & \raggedright{Levels already achieved} & \raggedright{Levels being attempted} & Selected Major \\
	\hline
	\hline
	\raggedright{\studA} & \sidA & X & X & \majA \\
	\raggedright{\studB} & \sidB & X & ?? & \majB \\
	\raggedright{\studC} & \sidC & X & A & \majC \\
	\raggedright{\studD} & \sidD & X & ?? & \majD \\
	\hline
\end{tabular}
\thispagestyle{empty}
\end{titlepage}
\pagenumbering{arabic}


%=======================================================================================

\tableofcontents


%=======================================================================================

\newpage
\section{Teamwork}
\label{sect-team}


\subsection{Developing industry skills}

To be trained/assisted by people who are well versed in new areas you are interested in proves to be an invaluable means of enriching your learning, given explanations are tailored to the individual and far more engaging than online courses. These people are inclusive of your superiors, coworkers or other members of your network. This approach is most appropriate once a basic understanding of the topic has been attained and you wish to master it whilst challenging yourself by developing complex projects and solving advanced problems. In this way, help is individualised and readily available. For instance, taking initiative to request and negotiate with a co-worker experienced in the field of interest for assistance - not only do they know real world applications, but also offer extensive knowledge to deepen your own understanding.

Online study courses are most appropriate for when someone wants to gain a comprehensive understanding of the programming language or topic in general. They offer the learner the opportunity to develop employable skills through the convenience of an online interface and even gain qualifications. They do however require an extensive amount of time and dedication to complete and therefore, may not be suitable for beginner programmers looking to extract quick and easy solutions as the information in these courses is often embedded amongst modules, or behind other boundaries such as having to complete previous tasks. Thus the online study courses are suitable for those willing to undertake a complete overview of a program.

Applicable coding in conjunction with google (i.e. stack overflow) is another effective approach that all programmers despite their respective field or level should use for their continuous learning. Applicable coding is absolutely necessary if a programmer strives to develop new skills and proficiency in arising programming languages. The approach enables a more advanced level of skill honing problem-solving and analysis skills to debug errors successfully (via research and google) and develop logically complex programs. Applicable coding serves as an effective method to ease into and understand all the nuanced syntaxes of a programming language. However, the method usually takes an extensive amount of time to work at before a programmer could see success for employability qualifications. Thus, applicable coding in conjunction with google / stack overflow is an essential part of continuous learning for programmers to hone their skills within their respective fields. 

Watching educational YouTube videos is a useful approach for learning a new tool or language. Coding tutorials and other similar videos give a great foundation for understanding the workflow of the program. This can be a helpful place to start when the tool is completely foreign, as you get to see just what it looks like to use the tool in an effective way. The downside to this approach is that learners can get stuck just watching people use the tool, rather than practically using the tool themselves. The best way to gain proficiency in a skill is to acquire experience doing it. Therefore, if a programmer is neglecting to combine practical applications with the YouTube tutorials they watch, they will fail to truly make any progress in their learning and proficiency.

A final approach that can be brought to continual learning is reading the official documentation for whichever tool is being learned. While documentation can be an effective tool as a reference to understand the framework of a program, it is not very practical for learning new languages when used on its own. Programming documentation often goes into much more detail than is necessary for a basic understanding, and can be hard to decipher due to the jargon being used. It is difficult to master reading on its own, hence it is best used alongside other resources or when encountering problems in implementation. Overall, reading programming documentation on its own is not a very effective approach to learning a programming language or tool, but can be useful if a learner gets stuck on their journey.


\subsection{Submission 1 contribution overview}

After initially receiving the group project, the team discussed various strategies to evenly split the tasks presented within the Teamwork section 1.1. We concluded it was best to brainstorm five of the approaches to continual learning together, and then assign each member to complete one. Given our team comprises four members, we took initiative to complete the remaining approach during a group meeting. As for individual level A attempts, all members were able to complete their section and pull/commit changes successfully. Within the final group meeting, we finalised and ensured the TEX file contained all necessary changes. Overall, there were equal contributions and a productive work flow between members.

\subsection{Submission 2 contribution overview}

As above, but completed for submission 2

\subsection{Submission 3 contribution overview}

As above, but completed for submission 3


%=======================================================================================

\newpage
\section{Level A: Basic Skills}

Level A focuses on basic technical skills (related to \LaTeX\ and Git) and the types of skills used in different computing jobs. Each member of the team should individually complete their subsection below. You should begin by allocating to each team member a different major to focus on (i.e. one of: Computer Science; Data Science; Software Development; Cyber Security). \textit{If you have a fifth member, then your tutor will suggest a fifth topic to cover}. This allocation should be specified above (see lines 36-55 in the LaTeX file).

You should then begin by looking at the list of skills identified within SFIA (Skills Framework for the Information Age)~\cite{sfia}. Then select two skills from the complete list:
\begin{enumerate}
	\item The skill in which you believe you are currently the strongest and which is relevant to the major you have selected. You should explain why this skill is necessary for that major, and what evidence you currently have that demonstrates your strength in this area. (Target = 200-400 words).
	\item The skill in which you believe you are currently the weakest but which is important to the major you have selected. You should explain why this skill is necessary for that major, and what you can do to improve your capability in this area. (Target = 200-400 words).
\end{enumerate}

You will need to integrate your information into the shared collaborative LaTeX document and compile the result.

\subsection{Skills for \majA: \studA}

-

\subsection{Skills for \majB: \studB}

Your text goes here

\subsection{Skills for \majC: \studC}

1. The skill that I believe I am strongest in and is also majorly relevant to the software development field is testing. Testing refers to investigating products, systems and services to assess whether the specified or unspecified requirements and characteristics are met. It involves the use of manual testing or automated testing where test cases are created defining the test conditions for a given requirement. Moreover, test outcomes must also be recorded and analysed to validate the program and improve upon errors. The skill in testing is fundamental to the nature of software development involving the process of planning, designing ,creating, testing and supporting software products. Software development as a whole works to conceive a successful product that guarantees quality assurance in which testing is essential for meeting these standards. Thus, expertise in testing is required to determine whether the output of a software product meets the benchmark and assure that the product is up to standard for consumers. I believe that I am most well-versed in testing given my experience in school to create software projects via an agile approach requiring methods of testing such as black box testing where I had designed a test case to isolate a specific function comparing the expected to actual output. Moreover, I had undergone work experience that required testing for products in development using an array of tools and testing approaches to further analyse and suggest improvements upon a more quality assured product. 

2. Application support is another communicational skill that is fundamental to the software development field and a skill that I believe to be the weakest in, given my inconsistency in providing information and unorganised work-life balance to routinely schedule for maintenance of products. Application support involves the deliverance of management, technical and administrative services to ensure a successful live application. Application support is a necessity in the software development cycle as a multitude of factors for a high-quality and reliable product need continual assistance post the development stage in order to meet a successful live application. For example, guidance and training for the users following new software releases, monitoring application performance, updating documentation and capturing user feedback are all application support skills that are a must-needed skill for every software developer in order to successfully produce a reliable product. Thus, Application support is by far one of the most important skills needed in order to become a successful software developer. However, I believe this to be one of my weakest skills due to the lack of experience in handling support post development as I have no experience in developing a live application (i.e. cloud-based) thus, cannot deem myself sufficient enough to deliver management, technical or administrative services for any customers. The most appropriate way to improve my capabilities in this area is to look for software development internships in my 3rd or 4th year and learn how to routinely schedule my assistance to customers for live-applications that have already been developed in these companies. 

\subsection{Skills for \majD: \studD}

Your text goes here


%=======================================================================================

\newpage
\section{Level B: Tools}

Level B focuses on exploration of key tools used within professional computing employment. All companies make use of a range of technologies and tools (often as part of a tech stack). These tools might be implementation languages; design tools; data analysis tools; collaboration technologies, etc. Each student should identify two tools that are widely used in industry and which relate to the major you are focusing on for this project. You should then describe:
\begin{enumerate}
	\item The main functionality of those tools;
	\item The ways in which those tools are used;
	\item Any weaknesses or limitations of those tools.
\end{enumerate}

As examples (which you shouldn't now use): Computer Science: eclipse; Software Development: github; Cyber Security: Wireshark; Data Science: Hadoop.

Note also that no two students in the same tutorial should choose the same tools, so your tutor will maintain a list of those that have already been selected. You should therefore check this list and then confirm your choice with your tutor prior to researching your proposed tools and spending time writing about them. (Target = $\sim$200-400 words per tool).

Also, in order to achieve level B each student needs to be able to demonstrate capability with git and compilation of LaTeX documents from the command line. To demonstrate this, your team (or at least those members who are aiming to attempt level B) should do the following:
\begin{enumerate}
	\item Select one member to:
	\begin{enumerate}
		\item Create a local github repository for the project. This repository should contain the main LaTeX documents, as well as a subdirectory called ''screengrabs'';
		\item Create a repository on github for the project;
		\item Connect your local repository to the remote github repo;
		\item Push your local repository contents to the remote repo;
		\item Add all team members (and your tutor and unit coordinator) as members to the remote repo;
	\end{enumerate}
	\item Each additional group member should then clone the remote repo;
	\item Each member aiming to achieve level B should then be able to use the remote repo (and pushing and pulling changes) to demonstrate collaborative editing of the LaTeX documents.
	\item And each member aiming to achieve level B should also do a screengrab (or multiple screengrabs) showing their local successful compilation, on the command line, of the final LaTeX document. This should be added to the screengrabs folder in your local repo and then pushed to the remote repo so that your tutor can view it.
\end{enumerate}

\subsection{Tools for \majA: \studA}

Your text goes here

\subsection{Tools for \majB: \studB}

Your text goes here

\subsection{Tools for \majC: \studC}

Your text goes here

\subsection{Tools for \majD: \studD}

Your text goes here


%=======================================================================================

\newpage
\section{Level C: Advanced Skills}

Level C focuses on more advanced technical skills in \LaTeX\ and Git.

The following is a list of advanced Git and \LaTeX\ skills/features. Each student in your team should select a different pair of items from each list (e.g. you might choose "Resetting and Tags" from the git list, and "Cross-referencing and Custom commands" from the LateX list). You then need to demonstrate actual use of each item (either through activity in Git, or through including items in this report). (Target = $\sim$100-200 words per student for each feature).
\begin{itemize}
	\item Git
	\begin{itemize}
		\item Rebasing and Ignoring files
		\item Forking and Special files
		\item Resetting and Tags
		\item Reverting and Automated merges
		\item Hooks and Tags
	\end{itemize}
	\item \LaTeX\ 
	\begin{itemize}
		\item Cross-referencing and Custom commands
		\item Footnotes/margin notes and creating new environments
		\item Floating figures and editing style sheets
		\item Graphics and advanced mathematical equations
		\item Macros and hyperlinks
	\end{itemize}
\end{itemize}

\subsection{Advanced skills: \studA}

Explain your use of the advanced Git and \LaTeX\ features. 

\subsection{Advanced skills: \studB}

Explain your use of the advanced Git and \LaTeX\ features. 

\subsection{Advanced skills: \studC}

Explain your use of the advanced Git and \LaTeX\ features. 

\subsection{Advanced skills: \studD}

Explain your use of the advanced Git and \LaTeX\ features. 



%=======================================================================================

\newpage
\section{Level D: Evolution of skills}

Level D focuses on understanding how professional practice might evolve in the future. Most students in this unit are likely to be at or near the start of your degree, and so it might be anywhere from 3 to 5 years before you really start working in industry full-time -- and the technology and ways in which people use them can change significantly in that time. 

Your answer to this section you should address the following (Target = $\sim$500 words):
\begin{enumerate}
	\item Describe what you believe will be the biggest change in the next 5 years in the tools or technologies that are being actively used in industry practice (in your selected major);
	\item Revisit the SFIA framework~\cite{sfia} from level A, and identify the one skill that you believe will have the biggest increase in terms of importance over the next 5 years. You should justify your choice.
\end{enumerate}


\subsection{Evolution of \majA: \studA}

Your text goes here

\subsection{Evolution of \majB: \studB}

Your text goes here

\subsection{Evolution of \majC: \studC}

Your text goes here

\subsection{Evolution of \majD: \studD}

Your text goes here



%=======================================================================================

\newpage

\bibliographystyle{ieeetran}
\bibliography{main}

\end{document}
\end{report}
